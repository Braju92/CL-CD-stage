\begin{quote}
{\bfseries{Project\+:}}
\begin{DoxyItemize}
\item Creare una pipeline Git\+Lab che utilizzi un immagine docker in grado di analizzare del codice fornito e produrre la documentazione, impaginando poi l\textquotesingle{}output su una pagina statica seguendo lo standard di doxygen.
\end{DoxyItemize}

{\bfseries{Tools\+:}}


\begin{DoxyItemize}
\item doxygen\+: produce documentazione analizzando il codice.
\item doxypypy\+: traduce la documentazione python ottenuta tramite doxygen in uno standard conforme con quello fornito per gli altri linguaggi.
\item Dockerfile\+: documento per la creazione di un immagine che contenga tutto il necessario; python, doxygen, doxypypy.
\item {\bfseries{Instructions\+:}}
\begin{DoxyItemize}
\item install doxygen
\item install pip
\item install doxypypy
\item install doxygen-\/gui
\item install graphviz (per le dipendenze di Dot)
\item crea e sposta py\+\_\+filter nel \$\+PATH 
\begin{DoxyCode}{0}
\DoxyCodeLine{\#!/bin/bash}
\DoxyCodeLine{doxypypy -\/a -\/c \$1}

\end{DoxyCode}

\item copy doxygen-\/awesome-\/css 
\begin{DoxyCode}{0}
\DoxyCodeLine{make install}

\end{DoxyCode}

\item use doxygen -\/g config\+\_\+file to create doxygen config file
\item set FILTER\+\_\+\+PATTERNS = $\ast$.py=py\+\_\+filter in doxygen config file to run Python code through doxypypy.
\item use doxygen config\+\_\+file to generate documentation
\end{DoxyItemize}
\item {\bfseries{more at \href{https://www.doxygen.nl/manual/doxygen_usage.html}{\texttt{ https\+://www.\+doxygen.\+nl/manual/doxygen\+\_\+usage.\+html}}}} 
\end{DoxyItemize}\end{quote}
